\documentclass[12pt,oneside]{article}
\usepackage{listings}
\usepackage[ruled]{algorithm2e}
\usepackage[czech]{babel}


\usepackage[a4paper,margin=2.5cm,head=0.5cm,foot=0.5cm]{geometry}
\usepackage{amsmath}
\usepackage{amsfonts}
\usepackage{amssymb}

\title{Domácí úkol k cvičení číslo 3}
\author{}

\begin{document}
	\maketitle
	\section{Rozhodnětě a \underline{zdůvodněte}, jestli platí následující tvrzení:}
		\begin{enumerate}
		\item Pro funkci $f(n) = 8n^2 + 7n + 6$ platí:
		\begin{align}
			f(n) \in \Omega(n^2), \\
			f(n) \in \Theta(n^2).
		\end{align}
		\item Pro funkci $g(n) = 4n\log(8n) + n + 10$ platí:
		\begin{align}
			g(n) \in O(n^2), \\
			g(n) \in \Theta(n\log n).
		\end{align}
		\item Pro funkci $h(n) = 3n^2 + 10^4n+  \pi $ platí:
		\begin{align}
			h(n) \in O(n^3) ,\\
			h(n) \in \Theta(n^3).
		\end{align}
	\end{enumerate}
	Za dostatečné zdůvodnění se považuje výpočet odpovídající limity nebo ukázka toho, že lze nalézt $c_1, c_2$ a $n_0$ v definicích $\Omega, \Theta, O$, viz přednáška a doporučená literatura.
	Dostatečné zdůvodnění by mělo obsahovat komentář v přirozeném jazyce, alespoň na úrovni: ,,Tvrzení platí/neplatí protože:\dots``
	
	\section{Určete, která z následujících funkcí roste rychleji:}
	Doporučeno je postupovat výpočtem limit, případné invenci a alternativním postupům se ovšem meze nekladou, dokud jsou správné.
	\begin{enumerate}
		\item $n^2$ nebo $n^3$
		\item $\sqrt{n}$ nebo $\log_2(n)$
		\item $n^2$ nebo $2^n$
		\item $n^4$ nebo $3n^3 + 3^3n^2 + 3^{3^3}n + 3^{3^{3^3}}$
	\end{enumerate}
	
	\section{Určete a zdůvodněte jaká je složitost daného algoritmu}
	Určete počet operací násobení, dělení, sčítaní a odečítání nutných pro běh algoritmu, nebo alespoň udělejte jejich řádový odhad.
	Předpokládejme, že cena všech operací násobení, sčítání, odečítání a dělení je stejná, a že je nezávislá na velikosti čísel.
	Dále se nebudeme trápit dělením nulou.
	Zápis \verb|for| cyklu v algoritmu je myšlený tak, že se začíná v dolní mezi a jde se do horní meze včetně.
	\begin{algorithm}[h]
		\caption{What does this do?}
		\DontPrintSemicolon
		//Input: $n\times (n+1)$ matrix $A[0\dots n-1; 0 \dots n]$ of real numbers \;
		\For{$i=0, \dots, n-2$}{
			\For{$j=i+1, \dots, n-1$}{
				\For{$k = i, \dots, n$}{
					$A[j, k] = A[j, k] - A[i, k]*A[j, i]/ A[i, i] $\;
				}
			}
		}
		\Return {$A$}
		\end{algorithm}
	
	Jak by šlo algoritmus snadno zefektivnit?
	
	\section{Vyřešte rekurentní rovnici}
	Najděte funkci $T$ takovou, že $T(0) = 1$ pro všechna $n \in \mathbb{N}$ (všechna kladná celá čísla) platí
	\begin{align}
		T(n) = 2 + T(n-1).
	\end{align}

\end{document}